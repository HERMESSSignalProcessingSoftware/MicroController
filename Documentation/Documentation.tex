%\documentclass[
%	11pt,			% Standardschrtift
%	a4paper,		% Seitengroesse
%	oneside,		% einseitiger Druck
%	parskip=half,		% Standard Paragraphformatierung
%	DIV=4,			% führt die Satzspiegelberechnung neu aus
%	captions=nooneline,	%
%	tablecaptionabove,	% Tabellenüberschriften aktivieren
%	bibliography=totoc,	% Stil fuer Literaturangaben
%	bibtotocnumbered,	% Literaturverzeichnis ins Inhaltsverzeichnis
%	liststotocnumbered,	% Alle Listen ins Inhaltsverzeichnis
%	headinclude,
%	headsepline,		% 
%	1.6headlines,		%
%	]
%	{book}
\documentclass{scrreprt}
\title{HERMESS Measurment System Specification (MESS) II }
\subtitle{Documentation}
\author{Kai Robin Grimsmann, M.Eng.}


\PassOptionsToPackage{table}{xcolor} % Fuer xcolor (option clash vermeiden)
\usepackage{xcolor} % Farben

%\usepackage[letterspace=200]{microtype} % Mikrokerning fuer sehr exakte Schriftarten
					% Einheit: 1/1000 eines em

\usepackage[left=3.00cm, right=3.00cm, top=3.00cm, bottom=3.00cm]{geometry} % Seitengeometrie

\usepackage[sfdefault]{noto} % Standardschrift aendern

\usepackage[T1]{fontenc} % Encoding
\usepackage[utf8]{inputenc}
\usepackage[ngerman]{babel} % Woerterbuch

\usepackage{sourcecodepro} % Standardschrift fuer Listings

\usepackage{listings} % Listings formatieren

\usepackage[german,algochapter]{algorithm2e} % Algorithmen
\SetAlFnt{\sffamily\small} %setzt den Font für die Algorithmen
\newcommand{\myAlgoFont}[1]{\small\textbf{\sffamily{#1}}}
\SetKwSty{myAlgoFont}
%Für algorithm2e Package die Caption setzen
\SetAlCapFnt{\sffamily\footnotesize}
\SetAlCapNameFnt{\sffamily\footnotesize}


\usepackage{csquotes}
\usepackage[justification=RaggedRight, singlelinecheck=false]{caption}
\usepackage{lstlinebgrd}

\usepackage{enumitem} % Aufzaehlungen mit eigenem Gliederungspunkt

\bibliographystyle{is-unsrt}

\usepackage[
    backend=biber,
    %style=is-unsrt,
    sortlocale=de_DE,
    natbib=true,
    url=false, 
    doi=true,
    eprint=false,
    backref=true %% In den Literaturangaben anzeigen, an welchen Stellen/Seiten das Zitat gesetzt ist
]{biblatex}
\addbibresource{main.bib} 
\usepackage{hyperref}

\usepackage{amsmath}

%%% In den Literaturangaben anzeigen, an welchen Stellen/Seiten das Zitat gesetzt ist
\DefineBibliographyStrings{german}{%
  backrefpage = {Seite},% originally "cited on page"
  backrefpages = {Seiten},% originally "cited on pages"
}


\usepackage{multicol}
\setlength{\columnsep}{1cm}

%\usepackage[singlespacing]{setspace} % Zeilenabstand Fliesstext normal
\usepackage[onehalfspacing]{setspace} % Zeilenabstand Fliesstext gedehnt
%\usepackage[doublespacing]{setspace} % Zeilenabstand Fliesstext doppelt gedehnt

\usepackage{adjustbox}

\usepackage{tabu} 
\usepackage{longtable}
\usepackage[table]{xcolor}
%\usepackage{float}
\usepackage{rotating}


%% Grafiken anzeigen:
\usepackage{graphicx}
%\usepackage[demo]{graphicx} % Grafiken im Schnellmodus (nur Boxen)

%% Beschriftungen neben Grafiken
\usepackage{floatrow}

%% Bilder als vorläufig markieren
\usepackage{overpic}
\newcommand{\draftImage}[2]{
	\begin{overpic}[#1]{#2}
		  \put(0,0){\includegraphics[#1]{#2}}
		    \put(15,5){\fontsize{30}{35}{\textbf{\begin{rotate}{45}VORLÄUFIG\end{rotate}}}}
	\end{overpic}
	}


\usepackage{nameref}


%% Randhinweise auch mit Grafiken
% Umgebung für Anmerkungen im Text
% (für den spacing-Befehl wird das setspace-Paket benötigt)
\usepackage{setspace}
% gelbe Hintergrundfarbe für Marginalie
\definecolor{sidebox_bg}{rgb}{1,1,0.4}
\newcommand{\sidenoteSingle}[1]{
        \hspace{0pt}%
        \marginpar{%
                \begin{spacing}{0.8}% Für geringeren Zeilenabstand
                        \fcolorbox{black}{sidebox_bg}{%
                                \parbox{.5\marginparwidth} {
                                        %\raggedright\sffamily\footnotesize{#1}%
                                        \sffamily\footnotesize{
						\begin{center}
							#1
							\vspace{2\baselineskip}
                                                \end{center}
					}%
                                }
                        }%
                \end{spacing}
        }
}

\newenvironment{androidversion}[1]%
    {\hspace{0pt}%    
     \marginpar{%
                %\vspace{1.8cm}
        \begin{spacing}{0.8}%
           %\fcolorbox{black}{sidebox_bg}{%
                \parbox{.5\marginparwidth}
                {
                        %\vspace{1.8cm}
                  \center
                        \includegraphics[width=.5\marginparwidth]{android_plush_robot.jpg}\
              \raggedright\sffamily
              \footnotesize{
                                                                \begin{center}
                                                                        Android~#1
                                                                        \vspace{2\baselineskip}
                                                                \end{center}                                   
                }%
                }          
        \end{spacing}
     }%
     }%
    {}%




%% Farbdefinitionen %%
\definecolor{color_c_comments}{rgb}{0.2, 0.7, 0.2}
\definecolor{color_c_keywords}{rgb}{1.0, 0.4, 0}
\definecolor{color_c_strings}{RGB}{180,116,19}
\definecolor{color_c_numbers}{rgb}{0.3, 0.3, 0.3}
\definecolor{color_bash_numbers}{rgb}{0.3, 0.3, 0.3}
\definecolor{color_lst_orange_light}{RGB}{255,250,241}


\definecolor{color_lst_gray_light}{RGB}{249,249,249}
\definecolor{color_keywords_background}{RGB}{255,238,205}
\definecolor{color_keywords_border}{rgb}{0.3, 0.3, 0.3}

\definecolor{tableHeader}{RGB}{255,238,205}
\definecolor{tableLineOne}{RGB}{245, 245, 245}
\definecolor{tableLineTwo}{RGB}{224, 224, 224}

\usepackage{tcolorbox} % Farbboxen
\newtcolorbox{importantbox}[1]{colback=red!5!white,
colframe=red!75!black,fonttitle=\bfseries,
title=#1}

%\newcommand{\keyword}[1]{\colorbox{color_keywords_background}{\texttt{#1}}} % Keywords ohne Box
\newcommand{\keyword}[1]{\adjustbox{margin=5pt 3pt, bgcolor=color_keywords_background, cfbox=orange 0.5pt 0pt, padding=2pt}{\texttt{#1}}} % Keywords mit Box
\newcommand{\srckeyword}[1]{\adjustbox{margin=5pt 3pt, bgcolor=color_lst_gray_light, cfbox=gray 0.5pt 0pt, padding=2pt}{\texttt{#1}}} % margin zuvor = 5pt 2pt 5pt 6pt

% Listings fuer Shellkommandos
\newcommand{\lstinputbash}[2]{
	\lstinputlisting[language=bash, #2, rulecolor=\color{gray},
		linebackgroundcolor={\ifodd\value{lstnumber}\color{color_lst_gray_light}\fi}
	]{#1}
}

\newcommand{\ignore}[1]{}
\newcommand{\pageWidth}{16.5cm}
%\newcommand{\quotes}[1]{``#1''}
\newcommand{\quotes}[1]{\glqq#1\grqq}
\newcommand{\techquotes}[1]{``#1''}

%% Tabellenstil mit Farbe
\newcommand{\tableHeaderStyle}{
    \rowfont{\leavevmode\color{black}\bfseries}
    \rowcolor{orange}
} 
\taburowcolors[2] 2{tableLineOne .. tableLineTwo}
\tabulinesep = ^2mm_1mm
\everyrow{\tabucline[.4mm  white]{}}

%% Listingstil fuer Makefiles
\lstdefinelanguage{Makefile}{
  sensitive=false, % keywords are not case-sensitive
  morecomment=[l]{\#}, % l is for line comment
  morestring=[b]" % defines that strings are enclosed in double quotes,
}

% Makefile aus einer Datei einbinden
\newcommand{\makefileinputlisting}[2]{
	\lstinputlisting[language=Makefile, #2, rulecolor=\color{gray},linebackgroundcolor={\ifodd\value{lstnumber}\color{color_lst_gray_light}\fi}]{#1}
}

\DeclareGraphicsExtensions{.png} % Grafikangabe

%% Einstellungen fuer Listings
\renewcommand\lstlistingname{Listing}
\renewcommand\lstlistlistingname{Programmcode-Verzeichnis}
\captionsetup{justification=RaggedRight,font={sf},labelsep=endash,labelfont=bf}

\lstset{
	basicstyle          =    \fontsize{9}{10}\ttfamily,
	commentstyle        =    \color{color_c_comments},
	frame               =    single,
	numbers             =    left,
	numbersep           =    5pt,
	numberstyle         =    \tiny\color{color_c_numbers},
	keywordstyle        =    \color{color_c_keywords},
	showspaces          =    false,
	showstringspaces    =    false,
	stringstyle         =    \color{color_c_strings},
	tabsize             =    2,
	breaklines          =    true,
	captionpos          =    b,
	belowskip           =    1em,
	abovecaptionskip    =    1em,
	aboveskip           =    3em,
	rulecolor           =    \color{orange},
	language            =    C,
	linebackgroundcolor =    {\ifodd\value{lstnumber}\color{color_lst_orange_light}\fi}
}

%% Nur zur Demo, kann in einer realen Ausarbeitung weg:
\usepackage{blindtext}


%% Makro zur Zeichnung von Gitternetzen
\newcount \myheight
\newcommand{\platz}[1]{
	\myheight=#1
	\advance \myheight by -1
	\multiply \myheight by 5
	\setlength{\unitlength}{1mm}
	\definecolor{grau}{gray}{0.8}
	\begin{picture}(175, \myheight)
		\color{grau}
		\thinlines
		\multiput(0,0)(5,0){36}{\line(0,1){\myheight}}
		\multiput(0,0)(0,5){#1}{\line(1,0){175}}
	\end{picture}
	}

 
\usepackage{acronym}

\begin{document}
\input{titlepage.tex}
\thispagestyle{empty}
\tableofcontents 
\pagebreak
%\section{Zweck dieses Dokumentes}
\chapter{Blockdiagramm}
\tikzset{
	sensor/.style={rectangle, draw, minimum size=1cm},
	component/.style={rectangle, draw, minimum size=4cm},
	vhdlc/.style={rectangle, draw, minimum size = 1cm},
	state/.style={circle, draw, minimum size=1cm},
	bigSensor/.style={rectangle, draw, minimum height = 3cm,  minimum width = 3cm}
}

\pgfdeclarelayer{background}
\pgfsetlayers{background,main}

\begin{landscape}
\begin{figure}[h]
\begin{tikzpicture}[very thick, black, node distance=2cm]
%\draw[step=1cm, gray, very thin] (0,0) grid (15,15);

\node[gray!50] (S1name) {\small{STAMP \# 1}};
\node[sensor, black, below of = S1name, node distance = 1cm] (S1SGR1) {\tiny{SGR 1}};
\node[sensor, black, below of = S1SGR1, node distance = 1.1cm] (S1pt100) {\tiny{PT100}};
\node[sensor, black, below of = S1pt100, node distance = 1.1cm] (S1SGR2) {\tiny{SGR 2}};

\node[rectangle, draw, gray!50, very thick, fit  = (S1name) (S1SGR2), inner sep=1mm, outer sep = 2mm] (S1all) {};

\coordinate[right of = S1SGR1] (hS1SGR1);
\coordinate[right of = S1pt100] (hS1pt100);
\coordinate[right of = S1SGR2] (hS1SGR2);

\node[gray!50, below of = S1SGR2, node distance = 1.5cm] (S2name) {\small{STAMP \# 2}};
\node[sensor, black, below of = S2name, node distance = 1cm] (S2SGR1) {\tiny{SGR 1}};
\node[sensor, black, below of = S2SGR1, node distance = 1.1cm] (S2pt100) {\tiny{PT100}};
\node[sensor, black, below of = S2pt100, node distance = 1.1cm] (S2SGR2) {\tiny{SGR 2}};

\node[rectangle, draw, gray!50, very thick, fit  = (S2name) (S2SGR2), inner sep=1mm, outer sep = 2mm] (S1all) {};

\coordinate[right of = S2SGR1] (hS2SGR1);
\coordinate[right of = S2pt100] (hS2pt100);
\coordinate[right of = S2SGR2] (hS2SGR2);

\node[gray!50, below of = S2SGR2, node distance = 1.5cm] (S3name) {\small{STAMP \# 3}};
\node[sensor, black, below of = S3name, node distance = 1cm] (S3SGR1) {\tiny{SGR 1}};
\node[sensor, black, below of = S3SGR1, node distance = 1.1cm] (S3pt100) {\tiny{PT100}};
\node[sensor, black, below of = S3pt100, node distance = 1.1cm] (S3SGR2) {\tiny{SGR 2}};

\node[rectangle, draw, gray!50, very thick, fit  = (S3name) (S3SGR2), inner sep=1mm, outer sep = 2mm] (S1all) {};

\coordinate[right of = S3SGR1] (hS3SGR1);
\coordinate[right of = S3pt100] (hS3pt100);
\coordinate[right of = S3SGR2] (hS3SGR2);


%SPU
\node[sensor, right of = S1SGR1, node distance = 2cm] (S1SGR1adc) {\tiny{ADS1147}}; 
\node[sensor, right of = S1pt100, node distance = 2cm] (S1pt100adc) {\tiny{ADS1147}};
\node[sensor, right of = S1SGR2, node distance = 2cm] (S1SGR2adc) {\tiny{ADS1147}};
\node[gray!50, above of = S1SGR1adc, node distance = 2cm] (S1fabName) {SPU}; 

\node[sensor, right of = S2SGR1, node distance = 2cm] (S2SGR1adc) {\tiny{ADS1147}}; 
\node[sensor, right of = S2pt100, node distance = 2cm] (S2pt100adc) {\tiny{ADS1147}};
\node[sensor, right of = S2SGR2, node distance = 2cm] (S2SGR2adc) {\tiny{ADS1147}};

\node[sensor, right of = S3SGR1, node distance = 2cm] (S3SGR1adc) {\tiny{ADS1147}}; 
\node[sensor, right of = S3pt100, node distance = 2cm] (S3pt100adc) {\tiny{ADS1147}};
\node[sensor, right of = S3SGR2, node distance = 2cm] (S3SGR2adc) {\tiny{ADS1147}};

%FPGA
%STAMP left
\node[black, right of = S1SGR1adc, node distance = 3cm] (F1S1name) {\small{c STAMP \#1}};
\node[vhdlc, below of = F1S1name, node distance = 1cm, xshift = -.4cm] (F1S1spi) {SPI};
\node[vhdlc, right of = F1S1spi, node distance = 1.5cm] (F1S1cu) {CU};
\node[rectangle, draw, below of = F1S1spi, black, node distance = 1cm, minimum width = 2.5cm, minimum height = .5cm] (F1S1rl) at ($(F1S1spi)!0.5!(F1S1cu)$) {ResetLogic};
\node[rectangle, draw, below of = F1S1rl, node distance = .75cm, minimum width = 2.5cm, minimum height = .5cm] (F1S1amba) {AMBA Conn. };
\node[rectangle, draw, black, very thick, fit  = (F1S1name) (F1S1rl)(F1S1amba), inner sep=2mm, outer sep = 2mm] (compStamp1) {};

\node[black, below = of F1S1name, yshift=-2.0cm] (F1S2name) {\small{c STAMP \#2}};
\node[vhdlc, below of = F1S2name, node distance = 1cm, xshift = -.4cm] (F1S2spi) {SPI};
\node[vhdlc, right of = F1S2spi, node distance = 1.5cm] (F1S2cu) {CU};
\node[rectangle, draw, below of = F1S2spi, black, node distance = 1cm, minimum width = 2.5cm, minimum height = .5cm] (F1S2rl) at ($(F1S2spi)!0.5!(F1S2cu)$) {ResetLogic};
\node[rectangle, draw, below of = F1S2rl, node distance = .75cm, minimum width = 2.5cm, minimum height = .5cm] (F1S2amba) {AMBA Conn. };
\node[rectangle, draw, black, very thick, fit  = (F1S2name) (F1S2rl) (F1S2amba), inner sep=2mm, outer sep = 2mm] (compStamp2) {};

\node[black, below  = of F1S2name, yshift=-2.0cm] (F1S3name) {\small{c STAMP \#3}};
\node[vhdlc, below of = F1S3name, node distance = 1cm, xshift = -.4cm] (F1S3spi) {SPI};
\node[vhdlc, right of = F1S3spi, node distance = 1.5cm] (F1S3cu) {CU};
\node[rectangle, draw, below of = F1S3spi, black, node distance = 1cm, minimum width = 2.5cm, minimum height = .5cm] (F1S3rl) at ($(F1S3spi)!0.5!(F1S3cu)$) {ResetLogic};
\node[rectangle, draw, below of = F1S3rl, node distance = .75cm, minimum width = 2.5cm, minimum height = .5cm] (F1S3amba) {AMBA Conn. };
\node[rectangle, draw, black, very thick, fit  = (F1S3name) (F1S3rl) (F1S3amba), inner sep=2mm, outer sep = 2mm] (compStamp3) {};

%Memory unit
\node[right  = of F1S1name] (F1MemName) {c Memory};
\node[vhdlc, below of =  F1MemName, node distance = 1.0cm, xshift=-.4cm] (F1MemSpi) {2x SPI};
\node[rectangle, draw, black, minimum height = 1cm, minimum width = 3cm, right of = F1MemSpi, node distance = 2.5cm] (F1MemCu){Controll Unit};
\node[rectangle, draw, black, minimum height = 1cm, minimum width = 4.6 cm, below of = F1MemSpi, node distance = 1.5cm, xshift = .5cm] (F1MemTG) at ($(F1MemSpi)!0.5!(F1MemCu)$) {Timestamp Generator};
\node[rectangle, draw, black, minimum height = 1cm, minimum width = 4.6cm, below of = F1MemTG, node distance = 1.5cm] (F1MemAmber) {AMBA Connector};


\node[rectangle, draw, black, very thick, fit=(F1MemName)(F1MemAmber), inner sep = 2mm, outer sep = 2mm] (compMemory){};

%SoC Controllunit
\node[rectangle, draw, black, very thick, minimum height = 1cm, minimum width = 4.6cm, below of = F1MemAmber , node distance = 3cm] (F1SoCControll) {AMBA Connector};
\node[above  = of F1SoCControll, xshift=-.4cm, yshift=-1.7cm] (F1SoCcuName) {c Global Control Unit};
\node[rectangle, draw, black, very thick, fit = (F1SoCcuName)(F1SoCControll), inner sep = 2mm, outer sep = 2mm] (compSoCCu) {};

%Watchdog
\node[rectangle, draw, black, very thick, minimum height = 1cm, minimum width = 4.6cm, below of = compSoCCu , node distance = 3.5cm] (F1Watchdog) {Watchdog CU};
\node[rectangle, draw, black, very thick, minimum height = 1cm, minimum width = 4.6cm, below of = F1Watchdog, node distance = 1.5cm] (F1WtdTimer) {Watchdog Timer};
\node[above of = F1Watchdog, xshift = -1.2cm, node distance = 1cm] (F1WdtName) {c Watchdog};
\node[rectangle, draw, black, very thick, fit = (F1WdtName)(F1WtdTimer), inner sep = 2mm, outer sep = 2mm] (compWatchdog) {};

%STAMP rechts
\node[black, right of = F1MemName, node distance = 6cm] (F1S4name) {\small{c STAMP \#4}};
\node[vhdlc, below of = F1S4name, node distance = 1cm, xshift = -.4cm] (F1S4cu) {CU};
\node[vhdlc, right of = F1S4cu, node distance = 1.5cm] (F1S4spi) {SPI};
\node[rectangle, draw, below of = F1S4spi, black, node distance = 1cm, minimum width = 2.5cm, minimum height = .5cm] (F1S4rl) at ($(F1S4spi)!0.5!(F1S4cu)$) {ResetLogic};
\node[rectangle, draw, below of = F1S4rl, node distance = .75cm, minimum width = 2.5cm, minimum height = .5cm] (F1S4amba) {AMBA Conn. };
\node[rectangle, draw, black, very thick, fit  = (F1S4name) (F1S4rl)(F1S4amba), inner sep=2mm, outer sep = 2mm] (compStamp4) {};

\node[black, below = of F1S4name, yshift = -2.0cm] (F1S5name) {\small{c STAMP \#5}};
\node[vhdlc, below of = F1S5name, node distance = 1cm, xshift = -.4cm] (F1S5cu) {CU};
\node[vhdlc, right of = F1S5cu, node distance = 1.5cm] (F1S5spi) {SPI};
\node[rectangle, draw, below of = F1S4spi, black, node distance = 1cm, minimum width = 2.5cm, minimum height = .5cm] (F1S5rl) at ($(F1S5spi)!0.5!(F1S5cu)$) {ResetLogic};
\node[rectangle, draw, below of = F1S5rl, node distance = .75cm, minimum width = 2.5cm, minimum height = .5cm] (F1S5amba) {AMBA Conn. };
\node[rectangle, draw, black, very thick, fit  = (F1S5name) (F1S5rl)(F1S5amba), inner sep=2mm, outer sep = 2mm] (compStamp5) {};

\node[black, below = of F1S5name, yshift=-2cm] (F1S6name) {\small{c STAMP \#6}};
\node[vhdlc, below of = F1S6name, node distance = 1cm, xshift = -.4cm] (F1S6cu) {CU};
\node[vhdlc, right of = F1S6cu, node distance = 1.5cm] (F1S6spi) {SPI};
\node[rectangle, draw, below of = F1S6spi, black, node distance = 1cm, minimum width = 2.5cm, minimum height = .5cm] (F1S6rl) at ($(F1S6spi)!0.5!(F1S6cu)$) {ResetLogic};
\node[rectangle, draw, below of = F1S6rl, node distance = .75cm, minimum width = 2.5cm, minimum height = .5cm] (F1S6amba) {AMBA Conn. };
\node[rectangle, draw, black, very thick, fit  = (F1S6name) (F1S6rl)(F1S6amba), inner sep=2mm, outer sep = 2mm] (compStamp6) {};

%Componente FPGA
\node[above of = F1S1name, node distance = 1cm, xshift=-0.8cm] (F1) {FPGA};
\node[rectangle, draw, black, very thick, fit  = (F1) (compStamp3) (compMemory)(compStamp4), inner sep=2mm, outer sep = 2mm] (F1all) {};

%SPU ADC rechts STAMP 4
\node[sensor, right  = of F1S4name] (S4SGR1adc) {\tiny{ADS1147}}; 
\node[sensor, below of = S4SGR1adc, node distance = 1.1cm] (S4pt100adc) {\tiny{ADS1147}};
\node[sensor, below of = S4pt100adc, node distance = 1.1cm] (S4SGR2adc) {\tiny{ADS1147}};

%SPU ADC rechts Sensoren Stamp 4
\node[sensor, black, right of = S4SGR1adc, node distance = 2.1cm] (S4SGR1) {\tiny{SGR 1}};
\node[sensor, black, below of = S4SGR1, node distance = 1.1cm] (S4pt100) {\tiny{PT100}};
\node[sensor, black, below of = S4pt100, node distance = 1.1cm] (S4SGR2) {\tiny{SGR 2}};
\node[gray!50, above of = S4SGR1, node distance = 1cm] (S4name) {\small{STAMP \# 4}};

\node[rectangle, draw, gray!50, very thick, fit  = (S4name) (S4SGR2), inner sep=1mm, outer sep = 2mm] (S4all) {};

\node[gray!50, below of = S4SGR2, node distance = 1.5cm] (S5name) {\small{STAMP \# 5}};
\node[sensor, black, below of = S5name, node distance = 1cm] (S5SGR1) {\tiny{SGR 1}};
\node[sensor, black, below of = S5SGR1, node distance = 1.1cm] (S5pt100) {\tiny{PT100}};
\node[sensor, black, below of = S5pt100, node distance = 1.1cm] (S5SGR2) {\tiny{SGR 2}};

\node[rectangle, draw, gray!50, very thick, fit  = (S5name) (S5SGR2), inner sep=1mm, outer sep = 2mm] (S5all) {};

\node[gray!50, below of = S5SGR2, node distance = 1.5cm] (S6name) {\small{STAMP \# 6}};
\node[sensor, black, below of = S6name, node distance = 1cm] (S6SGR1) {\tiny{SGR 1}};
\node[sensor, black, below of = S6SGR1, node distance = 1.1cm] (S6pt100) {\tiny{PT100}};
\node[sensor, black, below of = S6pt100, node distance = 1.1cm] (S6SGR2) {\tiny{SGR 2}};

\node[rectangle, draw, gray!50, very thick, fit  = (S6name) (S6SGR2), inner sep=1mm, outer sep = 2mm] (S6all) {};

% Restlichen ADCS
\node[sensor, left of = S5SGR1, node distance = 2.1cm] (S5SGR1adc) {\tiny{ADS1147}}; 
\node[sensor, left of = S5pt100, node distance = 2.1cm] (S5pt100adc) {\tiny{ADS1147}};
\node[sensor, left of = S5SGR2, node distance = 2.1cm] (S5SGR2adc) {\tiny{ADS1147}};

\node[sensor, left of = S6SGR1, node distance = 2.1cm] (S6SGR1adc) {\tiny{ADS1147}}; 
\node[sensor, left of = S6pt100, node distance = 2.1cm] (S6pt100adc) {\tiny{ADS1147}};
\node[sensor, left of = S6SGR2, node distance = 2.1cm] (S6SGR2adc) {\tiny{ADS1147}};

%SPU all
\node[rectangle, draw, gray!50, very thick, fit  = (S1fabName) (F1all) (S6SGR2adc), inner sep=2mm, outer sep = 2mm] (SPUall) {};

\path[black, draw, thick] (S1SGR1) -- (S1SGR1adc);
\path[black, draw, thick] (S1pt100) -- (S1pt100adc);
\path[black, draw, thick] (S1SGR2) -- (S1SGR2adc);

\path[black, draw, thick] (S2SGR1) -- (S2SGR1adc);
\path[black, draw, thick] (S2pt100) -- (S2pt100adc);
\path[black, draw, thick] (S2SGR2) -- (S2SGR2adc);

\path[black, draw, thick] (S3SGR1) -- (S3SGR1adc);
\path[black, draw, thick] (S3pt100) -- (S3pt100adc);
\path[black, draw, thick] (S3SGR2) -- (S3SGR2adc);

\path[black, draw, thick] (S4SGR1) -- (S4SGR1adc);
\path[black, draw, thick] (S4pt100) -- (S4pt100adc);
\path[black, draw, thick] (S4SGR2) -- (S4SGR2adc);

\path[black, draw, thick] (S5SGR1) -- (S5SGR1adc);
\path[black, draw, thick] (S5pt100) -- (S5pt100adc);
\path[black, draw, thick] (S5SGR2) -- (S5SGR2adc);

\path[black, draw, thick] (S6SGR1) -- (S6SGR1adc);
\path[black, draw, thick] (S6pt100) -- (S6pt100adc);
\path[black, draw, thick] (S6SGR2) -- (S6SGR2adc);

%SPI Path
\coordinate (tmpS1pt100adc) at ($(S1pt100adc)+(1cm,0cm)$);
\coordinate (tmpS1SGR2) at ($(S1SGR2adc)+(1cm,0cm)$);
\coordinate (tmpCS1) at ($(compStamp1)-(2.25cm,0cm)$);

\path[black, draw, line width = 3pt]  (S1SGR1adc) -- +(1cm, 0cm) -- (tmpS1pt100adc) -- (tmpS1SGR2) -- (S1SGR2adc);
\path[black, draw, line width = 3pt] (S1pt100adc) -- (tmpS1pt100adc);
\path[black, draw, line width = 3pt] (tmpCS1) -- +(.7cm,0cm);

%Stamp 2
\coordinate (tmpS2pt100adc) at ($(S2pt100adc)+(1cm,0cm)$);
\coordinate (tmpS2SGR2) at ($(S2SGR2adc)+(1cm,0cm)$);
\coordinate (tmpCS2) at ($(compStamp2)-(2.25cm,0cm)$);

\path[black, draw, line width = 3pt]  (S2SGR1adc) -- +(1cm, 0cm) -- (tmpS2pt100adc) -- (tmpS2SGR2) -- (S2SGR2adc);
\path[black, draw, line width = 3pt] (S2pt100adc) -- (tmpS2pt100adc);
\path[black, draw, line width = 3pt] (tmpCS2) -- +(.7cm, 0cm);

%Stamp 3
\coordinate (tmpS3pt100adc) at ($(S3pt100adc)+(1cm,0cm)$);
\coordinate (tmpS3SGR2) at ($(S3SGR2adc)+(1cm,0cm)$);
\coordinate (tmpCS3) at ($(compStamp3)-(2.25cm,0cm)$);

\path[black, draw, line width = 3pt]  (S3SGR1adc) -- +(1cm, 0cm) -- (tmpS3pt100adc) -- (tmpS3SGR2) -- (S3SGR2adc);
\path[black, draw, line width = 3pt] (S3pt100adc) -- (tmpS3pt100adc);
\path[black, draw, line width = 3pt] (tmpCS3) -- +(.7cm, 0cm);


%Stamp 4
\coordinate (tmpS4pt100adc) at ($(S4pt100adc)-(1cm,0cm)$);
\coordinate (tmpS4SGR2) at ($(S4SGR2adc)-(1cm,0cm)$);
\coordinate (tmpCS4) at ($(compStamp4)+(2.4cm,0cm)$);

\path[black, draw, line width = 3pt]  (S4SGR1adc) -- +(-1cm, 0cm) -- (tmpS4pt100adc) -- (tmpS4SGR2) -- (S4SGR2adc);
\path[black, draw, line width = 3pt] (S4pt100adc) -- (tmpS4pt100adc);
\path[black, draw, line width = 3pt] (tmpCS4) -- +(-.85cm, 0cm);

%Stamp 5
\coordinate (tmpS5pt100adc) at ($(S5pt100adc)-(1cm,0cm)$);
\coordinate (tmpS5SGR2) at ($(S5SGR2adc)-(1cm,0cm)$);
\coordinate (tmpCS5) at ($(compStamp5)+(2.4cm,0cm)$);

\path[black, draw, line width = 3pt]  (S5SGR1adc) -- +(-1cm, 0cm) -- (tmpS5pt100adc) -- (tmpS5SGR2) -- (S5SGR2adc);
\path[black, draw, line width = 3pt] (S5pt100adc) -- (tmpS5pt100adc);
\path[black, draw, line width = 3pt] (tmpCS5) -- +(-.85cm, 0cm);

%Stamp 6
\coordinate (tmpS6pt100adc) at ($(S6pt100adc)-(1cm,0cm)$);
\coordinate (tmpS6SGR2) at ($(S6SGR2adc)-(1cm,0cm)$);
\coordinate (tmpCS6) at ($(compStamp6)+(2.4cm,0cm)$);

\path[black, draw, line width = 3pt]  (S6SGR1adc) -- +(-1cm, 0cm) -- (tmpS6pt100adc) -- (tmpS6SGR2) -- (S6SGR2adc);
\path[black, draw, line width = 3pt] (S6pt100adc) -- (tmpS6pt100adc);
\path[black, draw, line width = 3pt] (tmpCS6) -- +(-.85cm, 0cm);

%Interne Busse

%AMBA BUS
\coordinate (AB1) at ($(compMemory) - (1.5cm, 2.65cm)$);
\coordinate (AB2) at ($(compSoCCu) - (1cm, 1.19cm)$);
\coordinate (AB3) at ($(compWatchdog) - (1.5cm, 1.9cm)$);
\coordinate (ABS1) at ($(F1S1amba) - (0cm, 0.5cm)$);
\coordinate (ABS2) at ($(F1S2amba) - (0cm, 0.5cm)$);
\coordinate (ABS3) at ($(F1S3amba) - (0cm, 0.5cm)$);
\coordinate (ABS4) at ($(F1S4amba) - (0cm, 0.5cm)$);
\coordinate (ABS5) at ($(F1S5amba) - (0cm, 0.5cm)$);
\coordinate (ABS6) at ($(F1S6amba) - (0cm, 0.5cm)$);

\path[red!50, draw, line width = 2pt] (AB1) -- +(0cm, -.23cm);
\path[red!50, draw, line width = 2pt] (AB2) -- +(0cm, -.35cm) ;
\path[red!50, draw, line width = 2pt] (AB3) -- + (0cm, -1cm);
\path[red!50, draw, line width = 2pt] (ABS1) -- +(0cm, -0.3cm) -- +(2.3cm, -0.3cm) -- + (2.3cm, -1.7cm) --+(3.6cm, -1.7cm); 
\path[red!50, draw, line width = 2pt] ($(ABS1) + (2.3cm, -1.7cm)$) -- +(0cm, -7.7cm); 
\path[red!50, draw, line width = 2pt] (ABS2) -- +(0cm, -0.3cm) -- +(4.15cm, -0.3cm); 
\path[red!50, draw, line width = 2pt] (ABS3) -- +(0cm, -0.3cm) -- +(3.6cm, -0.3cm); 
\path[red!50, draw, line width = 2pt] (ABS4) -- +(0cm, -0.3cm) -- +(-2.3cm, -0.3cm) -- +(-2.3cm, -1.7cm);
\path[red!50, draw, line width = 2pt] (ABS5) -- +(0cm, -0.3cm) -- +(-2.3cm, -0.3cm);
\path[red!50, draw, line width = 2pt] (ABS6) -- +(0cm, -0.3cm) -- +(-2.3cm, -0.3cm);
\path[red!50, draw, line width = 2pt] ($(ABS6) + (-2.3cm, -0.3cm)$) -- +(-5cm, 0cm); 
\path[red!50, draw, line width = 2pt] ($(ABS4) + (-2.3cm, -1.7cm)$) -- +(0cm, -7.7cm); 

% Reset Logic
\coordinate (rl6) at ($(compStamp6) - (1.78, 1cm)$);
\coordinate (rl5) at ($(compStamp5) - (1.78, 1cm)$);
\coordinate (rl4) at ($(compStamp4) - (1.78, 1cm)$);
\path[gray!70, draw, line width = 1.5pt] (rl6) -- +(0cm, -0.025cm) -- (rl4) -- +(0cm, 0.025cm);
\path[gray!70, draw, line width = 1.5pt] (rl6) -- + (.2cm, 0cm);
\path[gray!70, draw, line width = 1.5pt] (rl5) -- + (.2cm, 0cm);
\path[gray!70, draw, line width = 1.5pt] (rl4) -- + (.2cm, 0cm);

\coordinate (rl1) at ($(compStamp1) + (1.78, -1cm)$);
\coordinate (rl2) at ($(compStamp2) + (1.78, -1cm)$);
\coordinate (rl3) at ($(compStamp3) + (1.78, -1cm)$);
\path[gray!70, draw, line width = 1.5pt] (rl1) -- +(0cm, 0.025cm) -- (rl3) -- +(0cm,-0.025cm);
\path[gray!70, draw, line width = 1.5pt] (rl1) -- + (-.2cm, 0cm);
\path[gray!70, draw, line width = 1.5pt] (rl2) -- + (-.2cm, 0cm);
\path[gray!70, draw, line width = 1.5pt] (rl3) -- + (-.2cm, 0cm);

\coordinate (rlC1) at ($(rl3) + (0cm, 3.3cm)$);
\coordinate (rlC2) at ($(rl6) + (0cm, 3.3cm)$);
\coordinate (rlC3) at ($(compWatchdog) + (0cm, 2.2cm)$);
\path[gray!70, draw, line width = 1.5pt] (rlC1) -- (rlC2);
\path[gray!70, draw, line width = 1.5pt] (rlC3) -- + (0cm, -.3cm);


%Internet Datenbus
\coordinate (intCS6) at ($(compStamp6)-(1.55cm,0cm)$);
\coordinate (intCS66) at ($(intCS6)-(0.5cm,0cm)$);
\path[black, draw, line width = 3pt] (intCS6) -- +(-.5cm, 0cm);

\coordinate (intCS5) at ($(compStamp5)-(1.55cm,0cm)$);
\path[black, draw, line width = 3pt] (intCS5) -- +(-.5cm, 0cm);

\coordinate (intCS4) at ($(compStamp4)-(1.55cm,0cm)$);
\coordinate (intCS44) at ($(intCS4)-(.5cm,0cm)$);
\path[black, draw, line width = 3pt] (intCS4) -- +(-.5cm, 0cm);
\path[black, draw, line width = 3pt] (intCS44) -- +(0cm,.05cm) -- (intCS66) -- +(0cm, -0.05cm);

\coordinate (intCS1) at ($(compStamp1)+(1.55cm,0cm)$);
\coordinate (intCS11) at ($(intCS1)+(0.5cm,0cm)$);
\path[black, draw, line width = 3pt] (intCS1) -- +(.5cm, 0cm);

\coordinate (intCS2) at ($(compStamp2)+(1.55cm,0cm)$);
\path[black, draw, line width = 3pt] (intCS2) -- +(.5cm, 0cm);

\coordinate (intCS3) at ($(compStamp3)+(1.55cm,0cm)$);
\coordinate (intCS33) at ($(intCS3)+(.5cm,0cm)$);
\path[black, draw, line width = 3pt] (intCS3) -- +(.5cm, 0cm);
\path[black, draw, line width = 3pt] (intCS11) -- +(0cm,.05cm) -- (intCS33) -- +(0cm, -0.05cm);
\coordinate (memLeft) at ($(compMemory) - (2.55cm, 0cm)$);
\coordinate (memRight) at ($(compMemory) + (2.55cm, 0cm)$);
\path[black, draw, line width = 9pt] (memLeft) --+(-0.55cm,0cm);
\path[black, draw, line width = 9pt] (memRight) --+(0.4cm,0cm);
%Legende
\node[below of = S6SGR2adc, node distance = 1.4cm] (legende) {Legende}; 
\coordinate (l1) at ($(legende) -  (0cm, .5cm)$);
\coordinate (l2) at ($(legende) -  (0cm, 1cm)$);
\coordinate (l3) at ($(legende) -  (0cm, 1.5cm)$);
\path[red!70, draw, line width = 2pt] (l1) -- +(1cm, 0cm) node[right]{AMBA};
\path[gray!70, draw, line width = 1.5pt] (l2) -- +(1cm, 0cm) node[right]{Reset};
\path[black, draw, line width = 3pt] (l3) -- +(1cm, 0cm) node[right](intDB){Int. Datenbus};
\begin{pgfonlayer}{background}
	\node[rectangle, draw, black, fill=white, fit=(legende) (intDB), inner sep = 1mm, outer sep = 1mm, rounded corners = .5cm] (legendeall){};
\end{pgfonlayer}
\end{tikzpicture}
\caption{Systemübersicht}
\label{fig:blockGes}
\end{figure}
\end{landscape}
%Detailiertes blockdiagramm
\begin{landscape}
\begin{figure}[h]
\begin{tikzpicture}[very thick, black, node distance=6cm]
\node[gray!50] (S1name) {\small{STAMP \# 1}};
\node[bigSensor, black, below of = S1name, node distance = 2cm] (S1SGR1) {SGR 1};
\node[bigSensor, black, below of = S1SGR1, node distance = 3cm] (S1pt100) {PT100};
\node[bigSensor, black, below of = S1pt100, node distance = 3cm] (S1SGR2) {SGR 2};

\node[rectangle, draw, gray!50, very thick, fit  = (S1name) (S1SGR2), inner sep= 2mm, outer sep = 2mm] (S1all) {};

\node[bigSensor, right of = S1SGR1] (S1SGR1adc) {ADS1147}; 
\node[bigSensor, right of = S1pt100] (S1pt100adc) {ADS1147};
\node[bigSensor, right of = S1SGR2] (S1SGR2adc) {ADS1147};

\coordinate (data11) at ($(S1SGR1) + (1.5cm, 1cm)$);
\coordinate (data12) at ($(S1SGR1) + (1.5cm, -1cm)$);
\coordinate (data21) at ($(S1pt100) + (1.5cm, 1cm)$);
\coordinate (data22) at ($(S1pt100) + (1.5cm, -1cm)$);
\coordinate (data31) at ($(S1SGR2) + (1.5cm, 1cm)$);
\coordinate (data32) at ($(S1SGR2) + (1.5cm, -1cm)$);

\coordinate (data13) at ($(S1SGR1adc) - (1.5cm, 1cm)$);
\coordinate (data14) at ($(S1SGR1adc) - (1.5cm, -1cm)$);
\coordinate (data23) at ($(S1pt100adc) - (1.5cm, 1cm)$);
\coordinate (data24) at ($(S1pt100adc) - (1.5cm, -1cm)$);
\coordinate (data33) at ($(S1SGR2adc) - (1.5cm, 1cm)$);
\coordinate (data34) at ($(S1SGR2adc) - (1.5cm, -1cm)$);

\path[black, very thick, draw] (data11)--(data14);
\path[black, very thick, draw] (data12)--(data13);

\path[black, very thick, draw] (data21)--(data24);
\path[black, very thick, draw] (data22)--(data23);

\path[black, very thick, draw] (data31)--(data34);
\path[black, very thick, draw] (data32)--(data33);

\node[black, right of = S1SGR1adc, node distance = 8cm] (F1S1name) {\small{c STAMP \#1}};
\node[bigSensor, below of = F1S1name, node distance = 2cm] (F1S1spi) {SPI};
\node[bigSensor, right of = F1S1spi, node distance =3.5cm] (F1S1cu) {CU};
\node[rectangle, draw, below of = F1S1spi, black, node distance = 2.5cm, minimum width = 6.5cm, minimum height = 1cm] (F1S1rl) at ($(F1S1spi)!0.5!(F1S1cu)$) {ResetLogic};
\node[rectangle, draw, below of = F1S1rl, node distance = 1.5cm, minimum width = 6.5cm, minimum height = 1cm] (F1S1amba) {AMBA Connector};
\node[rectangle, draw, black, very thick, fit  = (F1S1name) (F1S1spi) (F1S1cu)(F1S1rl) (F1S1amba), inner sep=2mm, outer sep = 2mm] (compStamp1) {};

\coordinate (spirl1) at ($(F1S1spi) + (0cm, -1.5cm)$);
\coordinate (spirl2) at ($(F1S1rl) + (-1.5cm, 1cm)$);
\path[black, draw, ultra thick] (spirl1) -- (spirl2);

\coordinate (curl1) at ($(F1S1cu) + (0cm, -1.5cm)$);
\coordinate (curl2) at ($(F1S1rl) + (1.5cm, 1cm)$);
\path[black, draw, ultra thick] (curl1) -- (curl2);

\coordinate (pocADC11) at ($(S1SGR1adc) + (1.5cm, 1cm)$);
\coordinate (pocADC12) at ($(S1SGR1adc) + (1.5cm, -1cm)$);
\coordinate (pocADC13) at ($(S1SGR1adc) + (1.5cm, 0cm)$);

\coordinate (pocADC21) at ($(S1pt100adc) + (1.5cm, 1cm)$);
\coordinate (pocADC22) at ($(S1pt100adc) + (1.5cm, -1cm)$);
\coordinate (pocADC23) at ($(S1pt100adc) + (1.5cm, 0cm)$);

\coordinate (pocADC31) at ($(S1SGR2adc) + (1.5cm, 1cm)$);
\coordinate (pocADC32) at ($(S1SGR2adc) + (1.5cm, -1cm)$);
\coordinate (pocADC33) at ($(S1SGR2adc) + (1.5cm, 0cm)$);

\coordinate (pocSpi) at ($(F1S1spi) - (1.5cm, 0cm)$);
\coordinate (pocrl) at ($(F1S1rl) - (3cm, 0cm)$);
\coordinate (pocADC32) at ($(S1SGR2adc) + (1.5cm, -1cm)$);

\coordinate (pocCU) at ($(F1S1cu) + (0cm, 1.5cm)$);
%SPI 
\path[black, draw, line width = 3pt] (pocADC11) -- + (2cm, 0cm);
\path[black, draw, line width = 3pt] (pocADC21) -- + (2cm, 0cm);
\path[black, draw, line width = 3pt] (pocADC31) -- + (2cm, 0cm);
\path[black, draw, line width = 3pt] (pocADC11) -- + (2cm, 0cm) -- ($(pocADC31) + (2cm, -0.05cm)$); 
\path[black, draw, line width = 3pt] (pocSpi) -- + (-3cm, 0cm);

%DRDY 
\path[gray!70, draw, line width = 2pt] (pocADC13) -- +(2.5cm,0cm) ;
\path[gray!70, draw, line width = 2pt] (pocADC23) -- +(2.5cm,0cm);
\path[gray!70, draw, line width = 2pt] (pocADC33) -- +(2.5cm,0cm);
\path[gray!70, draw, line width = 2pt] (pocADC13) -- +(2.5cm,0cm) --  ($(pocADC33) + (2.5cm, -0.04cm)$);
\path[gray!70, draw, line width = 2pt] ($(pocADC13) +(2.5cm,0cm)$) -- ($(pocADC13) + (2.5cm, 2cm)$) -- ($(pocCU) + (0cm, 2.5cm)$) -- (pocCU); 
%Interner Datenbus
\coordinate (compDB) at ($(F1S1cu) + (1.5cm,0cm)$);
\path[blue, draw, line width = 5pt] (compDB) -- +(1.5cm,0cm) node[right]{48bit Datenbus};
%Write enable
%\coordinate (WEcomp) at ($(F1S1cu) + (1.5cm, -1cm)$);
%\path[blue, draw, line width = 2pt] (WEcomp) --+(1.5cm, 0cm) node[right]{Write Enable};
%Enable
\coordinate (Ecomp) at ($(compStamp1) + (3.45cm, -1cm)$);
\path[blue, draw, line width = 2pt] (Ecomp) -- +(1.3cm, 0cm) node[right]{Enable};
%clk
\coordinate (clkcomp) at ($(compStamp1) + (3.45cm, -2cm)$);
\path[blue, draw, line width = 2pt] (clkcomp) -- +(1.3cm, 0cm) node[right]{clk};
%clk
\coordinate (rstcomp) at ($(compStamp1) + (3.45cm, -3cm)$);
\path[blue, draw, line width = 2pt] (rstcomp) -- +(1.3cm, 0cm) node[right]{$\overline{reset}$};
%DataReady
\coordinate (drdycomp) at ($(compStamp1) + (3.45cm, 2cm)$);
\path[blue, draw, line width = 2pt] (drdycomp) -- +(1.3cm, 0cm) node[right]{$\overline{Data Ready}$};
%Amba bus
\coordinate (ambaC) at ($(compStamp1) + (3.45cm, 0cm)$);
\path[blue, draw, line width = 3pt] (ambaC) -- +(1.3cm, 0cm) node[right]{AMBA};
%legende
\coordinate (l) at ($(F1S1cu) + (2.5cm, 4.5cm)$);
\coordinate (l2) at ($(F1S1cu) + (2cm, 4cm)$);
\coordinate (l3) at ($(F1S1cu) + (2cm, 3.5cm)$);
\coordinate (l4) at ($(F1S1cu) + (2cm, 2.5cm)$);
\node at (l) (legende) {Legende};
\path[gray!70, draw, line width = 2pt] (l2)-- + (1cm,0cm) node[right](drdy){$\overline{DRDYn}$};
\path[black, draw, line width = 3pt] (l3) -- +(1cm, 0cm) node[right](intDB){SPI};
\node[below of = intDB, node distance = .5cm] (intSPI) {Din, Dout, Sclk, $\overline{CSn}$};
%\path[blue, draw, line width = 5pt] (l4) --+(1cm,0cm) node[right](databus){48bit Datenbus};
\begin{pgfonlayer}{background}
	\node[rectangle, draw, black, fill=white, fit=(legende)(drdy) (intDB) (intSPI), inner sep = 1mm, outer sep = 1mm, rounded corners = .5cm] (legendeall){};
\end{pgfonlayer}
\end{tikzpicture}
\caption{Messstelle, Digitalisierung, Datenverarbeitung}
\end{figure}

%Weiteres Bild
\begin{figure}[h]
\begin{tikzpicture}[very thick, black, node distance=6cm]
\node[black] (F1S1name) {\small{c STAMP \#1}};
\node[bigSensor, below of = F1S1name, node distance = 2cm] (F1S1spi) {SPI};
\node[bigSensor, right of = F1S1spi, node distance =3.5cm] (F1S1cu) {CU};
\node[rectangle, draw, below of = F1S1spi, black, node distance = 2.5cm, minimum width = 6.5cm, minimum height = 1cm] (F1S1rl) at ($(F1S1spi)!0.5!(F1S1cu)$) {ResetLogic};
\node[rectangle, draw, below of = F1S1rl, node distance = 1.5cm, minimum width = 6.5cm, minimum height = 1cm] (F1S1amba) {AMBA Connector};
\node[rectangle, draw, black, very thick, fit  = (F1S1name) (F1S1rl) (F1S1cu) (F1S1spi)(F1S1amba), inner sep=2mm, outer sep = 2mm] (compStamp1) {};

%Interner Datenbus
\coordinate (compDB) at ($(F1S1cu) + (1.5cm,0cm)$);
\path[blue, draw, line width = 5pt] (compDB) --node[above]{\small{48bit Datenbus}} +(4.75cm,0cm);
%DataReady
\coordinate (drdycomp) at ($(compStamp1) + (3.45cm, 2cm)$);
\path[blue, draw, line width = 2pt] (drdycomp) -- node[above]{\small{Data Ready}} +(4.55cm, 0cm);

%Write enable
%\coordinate (WEcomp) at ($(F1S1cu) + (1.5cm, -1cm)$);
%\path[blue, draw, line width = 2pt] (WEcomp) --  +(2.3cm, 0cm) --  node[above, rotate = 90]{\small{Write Enable}} +(2.3cm, -5cm);
%Enable
\coordinate (Ecomp) at ($(compStamp1) + (3.45cm, -1cm)$);
\path[blue, draw, line width = 2pt] (Ecomp) -- +(1.5cm, 0cm)  -- node[above, rotate = 90]{\small{Enable}} +(1.5cm, -4cm);
%clk
\coordinate (clkcomp) at ($(compStamp1) + (3.45cm, -2cm)$);
\path[blue, draw, line width = 2pt] (clkcomp) --  +(1.0cm, 0cm)  -- node[above, rotate = 90]{\small{clk}} +(1cm, -3cm);
%reset
\coordinate (rstcomp) at ($(compStamp1) + (3.45cm, -3cm)$);
\path[blue, draw, line width = 2pt] (rstcomp) -- +(0.5cm, 0cm)  -- node[above, rotate = 90]{\small{$\overline{reset}$}} +(0.5cm, -2cm);
%AMBA Stamp
\coordinate (S1amba) at ($(F1S1amba) -(0cm, .5cm)$);
\path[red!70, draw, line width = 3pt] (S1amba) --  node[above, rotate = 90]{AMBA} + (0cm, -2.3cm); 
\node[right  of = F1S1name, node distance = 11cm] (F1MemName) {c Memory};
\node[bigSensor, right of =  F1S1cu, node distance = 8.0cm] (F1MemSpi) {2x SPI};
\node[rectangle, draw, black, minimum height = 3cm, minimum width = 5cm, right of = F1MemSpi, node distance = 4cm] (F1MemCu){Controll Unit};
\node[rectangle, draw, black, minimum height = 1.5cm, minimum width = 8cm, below of = F1MemSpi, node distance = 2.5cm, xshift=.5cm] (F1MemTG) at ($(F1MemSpi)!0.5!(F1MemCu)$) {Timestamp Generator};
\node[rectangle, draw, black, minimum height = 1.5cm, minimum width = 8cm, below of = F1MemTG, node distance = 2cm] (F1MemAmber) {AMBA Connector};
\node[rectangle, draw, black, very thick, fit=(F1MemName)(F1MemAmber), inner sep = 2mm, outer sep = 2mm] (compMemory){};
\coordinate (amberbus) at ($(F1MemAmber) - (3cm, 0.75cm)$);
\path[red!70, draw, line width = 3pt] (amberbus) -- node[above, rotate = 90]{AMBA} +(0cm, -2cm);
%Signale zum Speicher
\coordinate (pocM1) at ($(F1MemSpi) + (-1cm, 2.5cm)$);
\path[black, draw, line width = 3pt] (pocM1) -- node[above, rotate = 90]{SPI \#1} + (0cm,1.75cm) ;
\coordinate (pocM11) at ($(F1MemSpi) + (0cm, 2.5cm)$);
\path[black, draw, line width = 3pt] (pocM11) -- node[above, rotate = 90]{SPI \#2} + (0cm,1.75cm) ;
\coordinate (pocM2) at ($(compMemory) + (0cm, 4.0cm)$);
\path[black, draw] (pocM2) -- node[above, rotate = 90]{$\overline{CS11}$} + (0cm,1.75cm) ;
\coordinate (pocM3) at ($(compMemory) + (1cm, 4.0cm)$);
\path[black, draw] (pocM3) -- node[above, rotate = 90]{$\overline{CS12}$} + (0cm,1.75cm) ;
\coordinate (pocM4) at ($(compMemory) + (2cm, 4.0cm)$);
\path[black, draw] (pocM4) -- node[above, rotate = 90]{$\overline{CS21}$} + (0cm,1.75cm) ;
\coordinate (pocM41) at ($(compMemory) + (3cm, 4.0cm)$);
\path[black, draw] (pocM41) -- node[above, rotate = 90]{$\overline{CS22}$} + (0cm,1.75cm) ;
%Allgemeine signale
\coordinate (poc1) at ($(compMemory) - (2cm, 4.0cm)$);
\path[black, draw] (poc1) -- node[above, rotate = 90]{clk} + (0cm,-1.75cm) ;
\coordinate (poc2) at ($(compMemory) - (1cm, 4.0cm)$);
\path[black, draw] (poc2) -- node[above, rotate = 90]{$\overline{reset}$} + (0cm,-1.75cm) ;
\coordinate (poc3) at ($(compMemory) - (0cm, 4.0cm)$);
\path[black, draw] (poc3) -- node[above, rotate = 90]{Enable} + (0cm,-1.75cm) ;
\end{tikzpicture}
\caption{Interface c Stamp zu c Memory}
\end{figure}
\end{landscape}
\chapter{Beschreibung}
\section{Komponente: STAMP}
Die STAMP Systemkomponente beschreibt einen Logikblock, der für die Kommunikation mit dem ADC 1147 zuständig ist. Um diese Kommunikation zu gewährleisten wird in jeder STAMP - Komponente ein SPI Core mit einer 8 - 16 Bit breiten  Busanbindung implementiert. Im vergleich zum bisherigen System würde die Anzahl der genutzten SPI - Systeme nicht erhöht werden. Die Systeme bedienen sich der FPGA inne liegenden Parallelität. 
\subsection{Beschreibung} 
\subsubsection{In- / Outputs}
\begin{itemize}
\item SPI [3]: Der klassische SPI Bus bestehend aus DIN, DOUT, SCLK.
\end{itemize}
\subsubsection{Input}
\begin{itemize}
\item $\overline{DRDY1,2,3}$ Interrupt vom ADC1147, fallende Flanke. Löst internen Process aus.  
\end{itemize}
\subsubsection{Outputs}
\begin{itemize}
\item $\overline{CS1,2,3} $ Chip Select der ADC 1147
\item $\overline{RESET1,2,3}$ Asynchroner Reset der ADC 1147
\item Data [48] Bus zur Memory Unit. 
\item Finished, steigende Flanke. Alle Daten wurden über SPI geholt und liegen an dem Data - Bus an. 
\end{itemize}
\subsection{Beschreibung STAMP.Cu ControllUnit} 
Die ControllUnit, kurz CU, steuert das Finished Signal. Ebenso erfolgt die Steuerung des SPI Cores über die ControllUnit.  Die CU registriert die fallende Flanke der ADC1147, auf Basis der fallenden Flanke des Interrupts wird der SPI Process gestartet. Hierbei wird der entsprechende Befehl an den ADC1147 gesendet und auf die empfangenden Daten gewartet. Ebenfalls stellt die CU sicher, dass der PT100 Sensor des STMAPs min. mit einer Frequenz von 20 Hz gelesen wird. Die Daten der DMS ADCs plus das Datum des PT100 ADCs werden in einem 48 bit breiten Bus parallel zur Verfügung gestellt. Bei dem deutlich langsamerem PT100 Datum wird das bestehenden Datum weiterverwendet. Nach dem alle drei Datums ermittelt wurden wechselt das Signal Finished auf High. \\\\
Ingesamt wird diese Konstruktion drei mal, für drei STAMP - Messstellen implementiert. 
\subsubsection{Datenformant auf data[48]}
Bit 47 - 24: PT 100 \\
Bit 23 - 16: DMS 2 \\
Bit 15 - 0: DMS 1 
\subsection{ResetLogic}
Wird durch den Watchdog eine Unregelmäßigkeit festgestellt wird das asynchrone Signal, IntReset ausgelöst. Dieses Signal wird durch den STAMP Core an den ADC1147 weitergeleitet. 
\section{Komponente: Watchdog}
\subsection{Beschreibung}
Der Watchdog implementiert einen internen Timer, wird dieser Timer ausgelöst, wird ein internes Reset - Signal generiert. Die Logik des Watchdogs lässt sich wie folgt beschreiben: \\\\
Wird durch einen STAMP binnen von 500ms kein Signal, Finished wechselt auf high, generiert, so wird der Reset für das Teilsystem ausgelöst. 
\\\\
Der Timer für das System muss entsprechend hoch dimensioniert werden. Eine entsprechend implementierter Prescaler ist vom Vorteil, aber nicht zwigend nötig.
\section{Komponente: Memory}
\subsection{Beschreibung}
Die Komponente hält 512 Byte Speicher vor, dieser Speicher entspricht der Größe zweier Speicher - Pages des Flashspeichers. \\\\
Ein Datemframe besteht aus 20 Bytes. Bytes 0 - 17: Daten der drei Messstellen. Bytes 18-19: 16 Bit CRC. \\\\
Ingesamt passen auf eine 256 Byte große Speicherseite 12 Datensätze. Ingesamt 16 Byte bleiben für Metadaten über. Die Metadaten halten mindestens den verweis auf den folgenden Datensatz in dem Format, insgesamt vier Byte: 2 Byte: Speichereinheits ID, Addresse auf der Einheit. \\\\
Des Weiteren können auf in den verbleibenden 12 Byte der Timestamp des ersten Datensatzes sowie der Timestamp des letzten Datensatzes gespeichert werden. \\\\
Timestamp, siehe Abschnitt \ref{timestamp}.
\subsubsection{Inputs}
\begin{itemize}
\item 3 x data [48] 
\item 3 x Finished 
\end{itemize}
\subsubsection{Output}
\begin{itemize}
\item 2 x SPI (DIN, DOUT, SCLK) je Flash double Die IC. 
\item 4 x $\overline{CS1.1, 1.2,  2.1, 2.2}$   
\end{itemize}
\section{Komponente: Timestampgenerator} 
\label{timestamp}
\subsection{Beschreibung}
Hauptaufgabe ist es das erstellen eines 32 bit langen Timestamps. 0 = SOE. Auflösung: 100$\mu s$. \\\\
$$\frac{Timestamps}{s} = \frac{1 s}{100 \mu s} = 10000$$
$$T = \frac{Aufl"osung}{\frac{Timestamps}{s}} = \frac{2^{32} - 1}{10\frac{k}{s}} = 423496s \approx 119h$$
Entsprechend der Auflösung sind ca. 119 Stunden Aufnahme mit dem System möglich. Zudem ermöglicht es die Auflösung von 100 $\mu s$ der Unterscheidun, von wann der Eintag mit Referenz zu T = 0.
\subsubsection{Output}
\begin{itemize}
\item Timestamp [32] Der ermittelte Timestamp.
\end{itemize}
\section{Resoucesbedarf, geschätzt}
\begin{table}[H]
\caption{Geschätze Anzahl der LogicElemente LEs}
\begin{tabular}{| l | l | l | l |}  \hline 
\textbf{Name} & \textbf{LEs} & \textbf{Anzahl} &\textbf{Gesamt} \\ \hline
SPI & 300 & 4 & 1200 \\ \hline
CU & 400 & 3 & 1200 \\ \hline 
WDT & 300 & 1 & 300 \\ \hline
RESET & 200 & 1 & 200 \\ \hline
Memory & 1500 & 1 & 1500 \\ \hline
APB & 400 & 1 & 400 \\ \hline 
\textbf{Gesamt} & & &  \textbf{4800} \\ \hline
\end{tabular}
\end{table}
Ingesamt wird mit dem Design 48 \% des FPGAs ausgenutzt. Grundlage für die Zahl ist die Anzahl der Logikeinheiten des FPGAs. In diesem Fall ist diese mit 10k angenommen. 
\end{document}
