\usepackage{tikz}
\usepackage{xcolor} % Farben
\usepackage{color, colortbl}
\usetikzlibrary{decorations.pathmorphing}
\usetikzlibrary{decorations.pathreplacing}
\usetikzlibrary{arrows, decorations.markings}
\usepackage{pgfplots}
\usepackage{pstricks-add}
\usepackage{amsmath}
\usepackage{amssymb}
\usepackage{caption}
\usepackage{stix}
\usepackage{booktabs}
\usepackage{nonfloat}
\usepackage{multicol}
\usepackage{tabularx}
\usepackage{url}
\usepackage{svg}
\usepackage[onehalfspacing]{setspace}
\usepackage[sfdefault]{noto} % Standardschrift aendern
\usepackage[utf8]{inputenc}
\usepackage{float}
\usepackage[english]{babel} % Woerterbuch
%\usepackage[ngerman]{babel}
\definecolor{tableHeader}{RGB}{255,238,205}
\definecolor{white}{RGB}{255,255,255}
\usepackage[left=2.50cm, right=2.50cm, top=2.50cm, bottom=3.0cm]{geometry} % Seitengeometrie
\newcommand*\circled[1]{\tikz[baseline=(char.base)]{
		\node[shape=circle,draw,inner sep=1pt] (char) {#1};}}
\newcommand{\mline}[1]{\begin{tabular}{@{}l@{}}#1\end{tabular}}
\newcommand\myfigure[1]{%
	\medskip\noindent\begin{minipage}{\columnwidth}
		\centering%
		#1%
		%figure,caption, and label go here
	\end{minipage}\medskip}

%\usepackage[
%backend=biber,
%style=is-unsrt,
%sortlocale=de_DE,
%natbib=true,
%url=false, 
%doi=true,
%eprint=false,
%backref=false %% In den Literaturangaben anzeigen, an welchen Stellen/Seiten das Zitat gesetzt ist
%]{biblatex}
%\addbibresource{test.bib} 
\newcommand{\quotes}[1]{\glqq#1\grqq}