\newgeometry{
  left=1cm,
  right=1cm,
  top=1cm,
  bottom=1cm,
  bindingoffset=5mm
}
\thispagestyle{empty}
\begin{center}
	\bfseries{\Large{Bewertung der \thesisType arbeit}}\\
\end{center}
\begin{center}
	\bfseries{\thesisTitle} von \thesisAuthor\\
\end{center}
\begin{center}
	\thesisSupervisor , am \today
\end{center}

\vspace{2ex}

\noindent{Qualität der Lösung: \textbf{Arbeitsweise und Lösungsweg}}
\begin{table}[H]
	\begin{tabu} to \textwidth {X[3,l] X[1,l] | X[1,l]}
       \tableHeaderStyle
        Kriterium & Selbsteinschätzung & Bewertung \\
	Zeiteinteilung des Studierenden & & \\
	Methodische Arbeitsweise & & \\
	Problemchen führt zum wochenlangen Abtauchen & & \\
	Aufteilung in Arbeitspakete & & \\
\end{tabu}
\end{table}
\begin{center}
	\platz{6}
\end{center}


\noindent{Qualität der Lösung: \textbf{Inhalt}}
\begin{table}[H]
	\begin{tabu} to \textwidth {X[3,l] X[1,l] | X[1,l]}
       \tableHeaderStyle
        Kriterium & Selbsteinschätzung & Bewertung \\
	Wurde das Problem angemessen gelöst & & \\
	Wurde etwas neues erreicht & & \\
	Passt der Umfang der Lösung zur Arbeitszeit & & \\
	Hinweise auf offene/ungelöste Probleme & & \\
	Nachvollziehbarkeit, Diskussion der eigenen Lösung? & & \\
	Wie viele Behauptungen werden nicht belegt/nicht bewiesen? & & \\
\end{tabu}
\end{table}
\begin{center}
	\platz{6}
\end{center}


\noindent{Qualität der Lösung: \textbf{Dokumentation und Ausarbeitung}
\begin{table}[H]
	\begin{tabu} to \textwidth {X[3,l] X[1,l] | X[1,l]}
       \tableHeaderStyle
        Kriterium & Selbsteinschätzung & Bewertung \\
	Lesbarkeit, Sprachstil und Verständlichkeit & & \\
	Systematik und roter Faden & & \\
	Passende Beispiele und Bilder & & \\
	Korrekte Zitierweise, passende Zitate & & \\
\end{tabu}
\end{table}
\begin{center}
	\platz{6}
\end{center}
\newpage
\thispagestyle{empty}

\noindent{Qualität der Lösung: \textbf{Implementierung}
\begin{table}[H]
	\begin{tabu} to \textwidth {X[3,l] X[1,l] | X[1,l]}
       \tableHeaderStyle
        Kriterium & Selbsteinschätzung & Bewertung \\
	Programmierstil, Lesbarkeit & & \\
	Weiterverwendung, Integrationsfähigkeit & & \\
\end{tabu}
\end{table}
\begin{center}
	\platz{6}
\end{center}


\noindent\textbf{Schwierigkeitsgrad} der Aufgabe
\begin{table}[H]
	\begin{tabu} to \textwidth {X[3,l] X[1,l] | X[1,l]}
       \tableHeaderStyle
        Kriterium & Selbsteinschätzung & Bewertung \\
	Verhältnis theoretischer zu praktischer Teil (bei geg. Aufgabe) & & \\
	Welcher Teil der Lösung war vorgegeben & & \\
	Nutzung von Wissen aus Lehrveranstaltung & & \\
	Wie formal bzw. mathematisch ist die Lösung? & & \\
\end{tabu}
\end{table}
\begin{center}
	\platz{6}
\end{center}


\noindent\textbf{Selbständigkeit} bei der Bearbeitung
\begin{table}[H]
	\begin{tabu} to \textwidth {X[3,l] X[1,l] | X[1,l]}
       \tableHeaderStyle
        Kriterium & Selbsteinschätzung & Bewertung \\
	Recherche, Arbeit mit Literatur/Bibliothek & & \\
	Eigene oder vorgeschlagene Lösungswege & & \\
	Regelmäßige Treffen, regelmäßiger Report & & \\
	Betreuungsaufwand (nicht selbständiges Arbeiten) & & \\
\end{tabu}
\end{table}
\begin{center}
	\platz{14}
\end{center}


\restoregeometry
